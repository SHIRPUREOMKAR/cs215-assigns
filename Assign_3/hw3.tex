\title{Assignment 3: CS 215}
\author{}
\date{Due: 14th September before 11:55 pm}

\documentclass[11pt]{article}

\usepackage{amsmath}
\usepackage{amssymb,ulem}
\usepackage{hyperref,color}
%\usepackage{ulem}
\usepackage[margin=0.3in]{geometry}
\begin{document}
\maketitle

\textbf{All members of the group should work on all parts of the assignment. Copying across groups or from other sources is not allowed. We will adopt a \textbf{zero-tolerance policy} against any violation.}
\\
\\
\textbf{Submission instructions:} 
\begin{enumerate}
\item You should type out a report containing all the answers to the written problems in Word (with the equation editor) or using Latex, or write it neatly on paper and scan it. In either case, prepare a single pdf file. 
\item The report should contain names and roll numbers of all group members on the first page as a header.
\item Put the pdf file and the code for the programming parts all in one zip file. The pdf should contain the names and ID numbers of all students in the group within the header. The pdf file should also contain instructions for running your code. Name the zip file as follows: A2-IdNumberOfFirstStudent-IdNumberOfSecondStudent-IdNumberofThirdStudent.zip. (If you are doing the assignment alone, the name of the zip file is A2-IdNumber.zip, if there are two students it should be A2-IdNumberOfFirstStudent-IdNumberOfSecondStudent.zip). 
\item Upload the file on moodle BEFORE 11:55 pm on the due date. We will nevertheless allow and not penalize any submission until 10:00 am on the following day (i.e. 15th September). No assignments will be accepted thereafter. 
\item Note that only one student per group should upload their work on moodle, though all group members will receive grades. 
\item Please preserve a copy of all your work until the end of the semester. 
\end{enumerate}

\textbf{Questions:}
\begin{enumerate}
\item Consider a shelf containing $n$ books, each one with a distinct color. Let us suppose that you pick a book uniformly at random with replacement (i.e. you put the book back on the shelf after picking it) and independently of what was picked earlier. Let $X^{(n)}$ be the number of times you would need to pick a book in this fashion, such that you have chosen a book of each color at least once. We can write that $X^{(n)} = X_1 + X_2 + ...+ X_n$ where $X_i$ denotes the additional number of times you have to pick a book such that you move from having picked books of $i-1$ distinct colors to $i$ distinct colors. We wish to determine $E(X)$ and $Var(X)$. To this end, do as follows:
\begin{enumerate}
\item What is $X_1$? When books with $i-1$ distinct types of colors have been collected, what is the probability of picking a book with a different color (i.e. different from the previous $i-1$ colors)? \textsf{[3 points]}
\item Due to independence, $X_i$ is a geometric random variable. What is its parameter? Let $Z$ be a random variable for the trial number for the first head obtained in a sequence of independent Bernoulli trials with head probability $p$. Then $P(Z = k) = (1-p)^{k-1} p$ where $k = 1,2,3,...$, and $Z$ is said to be a geometric random variable with parameter $p$.  \textsf{[3 points]}
\item Show that the expected value of a geometric random variable with parameter $p$ is $1/p$. Derive the variance of a geometric random variable.  \textsf{[4+4=8 points]}
\item Hence derive $E(X^{(n)})$ for this problem.  \textsf{[3 points]}
\item Hence derive an upper bound on $Var(X^{(n)})$ for this problem.  You will need the inequality that the sum of reciprocals of squares of positive integers is upper bounded by $\pi^2/6$.  \textsf{[3 points]}
\item Plot a graph of $E(X^{(n)})$ versus $n$ for different $n$. If $E(X^{(n)}) = \Theta(f(n))$, what is $f(n)$? \textsf{[3+2=5 points]}
\end{enumerate}

\item \begin{enumerate}
\item A student is trying to design a procedure to generate a sample from a distribution function $F$, where $F$ is invertible. For this, (s)he generates a sample $u_i$ from a $[0,1]$ uniform distribution using the `rand' function of MATLAB, and computes $v_i = F^{-1}(u_i)$. This is repeated $n$ times for $i = 1 ... n$. Prove that the values $\{v_i\}_{i=1}^n$  follow the distribution $F$. \textsf{[6 points]}
\item Let $Y_1, Y_2, ..., Y_n$ represent data from a continuous distribution $F$. The empirical distribution function $F_e$ of these data is defined as $F_e(x) = \dfrac{\sum_{i=1}^n \mathbf{1}(Y_i \leq x)}{n}$ where $\mathbf{1}(z) = 1$ if the predicate $z$ is true and 0 otherwise. Now define $D = \textrm{max}_x | F_e(x)-F(x) |$. Also define $E = \textrm{max}_{0 \leq y \leq 1} \Big|\dfrac{\sum_{i=1}^n \mathbf{1}(U_i \leq y)}{n} - y \Big|$ where $U_1, U_2, ..., U_n$ represent data from a $[0,1]$ uniform distribution. Now prove that $P(E \geq d) = P(D \geq d)$. \emph{Briefly} explain what you think is the practical significance of this result in statistics. \textsf{[6+5=11 points]}
\end{enumerate}

\item 
\begin{enumerate}
\item In this exercise, we will perform maximum likelihood based plane fitting. Let the equation of the plane be $z = ax + by + c$. Let us suppose we have access to accurate $X$ and $Y$ coordinates of some $N$ points lying on the plane. We also have access to the $Z$ coordinates of these points, but those have been corrupted independently by noise from $\mathcal{N}(0,\sigma^2)$. Write down the log-likelihood function $\mathcal{L}$ to be maximized in order to determine $a,b,c$. Write down three linear equations corresponding to setting partial derivatives of $\mathcal{L}$ w.r.t. $a,b,c$ (respectively) to 0. Express these equations in matrix and vector form. \textsf{[3+4=7 points ]} \\

\item Repeat the previous part if $z$ had the form $z = a_1 x^2 + a_2 y^2 + a_3 xy + a_4 x + a_5 y + a_6$. Again, let us suppose we have access to accurate $X$ and $Y$ coordinates of some $N$ points lying on the plane. We also have access to the $Z$ coordinates of these points, but those have been corrupted independently by noise from $\mathcal{N}(0,\sigma^2)$. Write down the log-likelihood function $\mathcal{L}$ to be maximized in order to determine $a_1,a_2,...,a_6$. Write down linear equations corresponding to setting partial derivatives of $\mathcal{L}$ w.r.t. $a_1,a_2,...,a_6$ (respectively) to 0. Express these equations in matrix and vector form. \textsf{[4+4=8 points]}\\

\item Now write MATLAB code to solve this linear system for data consisting of XYZ coordinates of $N = 2000$ points, stored in the file `XYZ.txt' in the homework folder. Read the data using the MATLAB function `dlmwrite'. The data consist of $N$ rows, each containing the X,Y,Z coordinates of a point (in that order). What is the predicted equation of the plane? What is the predicted noise variance? State these in your report, and print them out via your code. \textsf{[10 points]}
\end{enumerate}

\item We have extensively seen parametric PDF estimation in class via maximum likelihood. In many situations, the family of the PDF is however unknown. Estimation under such a scenario is called nonparametric density estimation. We have studied one such technique in class, namely histogramming, and we also analyzed its rate of convergence. There is another popular technique for nonparametric density estimation. It is called KDE or Kernel density esitmation, the formula for which is given as $\hat{p}_n(x;\sigma) = \dfrac{\sum_{i=1}^n \exp{(-(x - x_i)^2/(2 \sigma^2))}}{n \sigma \sqrt{2 \pi}}$. Here $\hat{p_n}(x)$ is an estimate of the underlying probability density at value $x$, $\{x_i\}_{i=1}^n$ are the $n$ samples values, from which the unknown PDF is being estimated, and $\sigma$ is a bandwidth parameter (similar to a histogram bin-width parameter). The choice of the appropriate $\sigma$ is not very straightforward. We will implement one possible procedure to choose $\sigma$ - called cross-validation. For this, do as follows:
\begin{enumerate}
\item Use MATLAB to draw $n = 1000$ independent samples from $\mathcal{N}(0,16)$. We will use a random subset of 750 samples (set $T$) for building the PDF, and the remaining 250 as the validation set $V$. Note that $T$ and $V$ must be disjoint sets. 
\item In your report, write down an expression for the joint likelihood of the samples in $V$, based on the estimate of the PDF built from $T$ with bandwidth parameter $\sigma$. \textsf{[3 points]}
\item For different values of $\sigma$ from the set $\{0.001, 0.1, 0.2, 0.9, 1, 2, 3, 5, 10, 20, 100\}$, write MATLAB code to evaluate the log of the joint likelihood $LL$ of the samples in $V$, based on the estimate of the PDF built from $T$. Plot of a graph of $LL$ versus $\log \sigma$ and include it in your report. In the report, state which value of $\sigma$ yielded the best $LL$ value, and print it via your code as well.  This procedure is called cross-validation. For this best sigma, plot a graph of $\hat{p}_n(x;\sigma)$ for $x \in [-8:0.1:8]$ and overlay the graph of the true density on it, for the same values of $x$. Include this plot in your report. \textsf{[7 points]}
\item In this experiment, we know the ground truth pdf which we shall denote as $p(x)$.  So we can peek into it, in order to choose the best $\sigma$. This is impractical in actual experiments, but for now it will serve as a method of comparison. For each $\sigma$, write MATLAB code to evaluate $D = \sum_{x_i \in V} (p(x_i)-\hat{p}_n(x_i;\sigma))^2$. Plot of a graph of $D$ versus $\log \sigma$ and include it in the report. In the report, state which value of $\sigma$ yielded the best $D$ value, and also what was the $D$ value for the $\sigma$ parameter which yielded the best $LL$. For this best sigma, plot a graph of $\hat{p}_n(x;\sigma)$ for $x \in [-8:0.1:8]$ and overlay the graph of the true density on it, for the same values of $x$. Include this plot in your report. \textsf{[7 points]}
\item Now, suppose the set $T$ and $V$ were equal to each other. What happens to the cross-validation procedure, and why? Explain in the report. \textsf{[4+4=8 points]}
\end{enumerate}

\item Let $X$ be a real-valued random variable whose values lie from $a$ to $b$ always, where $a < b$. Then consider an intermediate result (called \textsf{IR}) that $E[e^{s(X-E[X])}] \leq e^{s^2(b-a)^2/8}$ where $s > 0$. Now, let $X_1, X_2,...,X_n$ be independent random variables such for every $i$, we have $X_i$ always lies in $[a_i, b_i]$ where $a_i < b_i$. Let $S_n = \sum_{i=1}^n X_i$. Derive an upper bound on $P(S_n - E[S_n] > t)$ in terms of $a_i, b_i, t$ using Markov's inequality and \textsf{IR}, and upon suitable elimination of $s$. Notice that  \textsf{IR} is an upper bound on the moment generating function of random variable $X$ with bounded values. We will now proceed to prove \textsf{IR} as follows:
\begin{enumerate}
\item Without loss of generality, we consider $E(X) = 0$, because $X$ can be replaced by $X-E(X)$ anyways. Hence, we consider $a \leq 0 \leq b$. The function $e^{sx}$ is a convex function of $x$, and hence a line segment joining two distinct points of the graph always lies above the graph of the function between the two points. Hence $e^{sx} \leq \dfrac{(b-x)e^{sa}}{b-a} + \dfrac{(x-a)e^{sb}}{b-a}$. 
\item Taking, expectation on both sides, prove that $E(e^{sx}) \leq e^{L(s(b-a))}$ where $L(h) = \dfrac{ha}{b-a} + \log(1 + (a-ae^h)/(b-a))$. 
\item Using Taylor's expansion and the result that $(x+y)/2 \geq \sqrt{xy}$ for real-valued $x,y$, prove that $L(h) \leq 1/4$ for all real-valued $h$.
\item Hence, conclude the proof (write the final, now somewhat obvious step).  
\end{enumerate} \textsf{[5 + (1+3+1) = 10 points]}

\end{enumerate}
\end{document}